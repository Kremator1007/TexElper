%this is latex 2e
\documentclass[12pt,a4paper]{article}
\usepackage{amsmath,amssymb,amsthm}
\usepackage[mathscr]{eucal}
\usepackage[T2A]{fontenc}
\usepackage[utf8]{inputenc}
\usepackage[russian]{babel}
\usepackage{fancyhdr}
% \usepackage{wrapfig}
% \RequirePackage[pdftex]{graphicx}

\usepackage{pgf,tikz,pgfplots}
\pgfplotsset{compat=1.9}
\usepackage{mathrsfs}
\usetikzlibrary{arrows}

% \usepackage{amsthm}

% \oddsidemargin=-10mm
% \textwidth=180mm
% \headheight=0pt
% \headsep=0pt
% \topmargin=-10mm
% \textheight=260mm

\sloppy
\righthyphenmin=2
\exhyphenpenalty=10000

\pagestyle{empty}

\def\ds{\displaystyle}
\def\ss{\scriptstyle}

\def\q#1.{{\bf #1.}}
\def\N{\mathbb N}
\def\Z{\mathbb Z}
\def\Q{\mathbb Q}
\def\R{\mathbb R}
\def\Pp{\mathbb P}
\def\C{{\mathbb C}}
\def\F{{\mathbb F}}
\def\eps{{\varepsilon}}
\def\vec{\overrightarrow}
\def\ov{\overline}
\def\mmax{\mathop{\rm max}\limits}
\newcommand{\Char}{{\rm char}}
\def\sleg#1#2{\left(\frac{#1}{#2}\right)}
\def\ord{\rm ord}
\def\arc#1{\buildrel\,\,\smile\over{#1}}
\DeclareRobustCommand{\No}{\ifmmode{\nfss@text{\textnumero}}\else\textnumero\fi}

\DeclareMathOperator{\Ker}{Ker}
\DeclareMathOperator{\Ima}{Im}
\DeclareMathOperator{\Int}{Int}
\DeclareMathOperator{\Cl}{Cl}

\newtheorem*{utv}{Утверждение}
\newtheorem*{thm}{Теорема}
\newtheorem*{rem}{Замечание}
\theoremstyle{definition}
\newtheorem{defin}{Определение}
\newtheorem*{ddefin}{Определение}

\fancypagestyle{firststyle} 
{
\fancyhead[L]{{\slshape {@Kusaka}}}
\fancyhead[R]{Летние задачи}
\renewcommand{\headrulewidth}{0.2 mm} 
\fancyfoot{} 
}

\begin{document}

\thispagestyle{firststyle}

\centerline{\bf Серия 1}

%\enlargethispage{1\baselineskip}

\q1. Докажите, что число $4^n(8k - 1)$ не представимо в виде суммы трех квадратов целых чисел.

\q2. В графе любое ребро инцедентно с вершиной, степень которой  не превосходит четырех. Какое наибольшее количество ребер может быть в таком графе, если вершин сорок?

\q3. Докажите неравенство $a^4 + b^4 + c^4 \geq abc(a + b + c)$

\q4. Вычислите $1 \cdot 2 \cdot 3 + 2 \cdot 3 \cdot 4 + \cdot \cdot \cdot + 2020 \cdot 2021 \cdot 2022$

\q5. Три вещественных числа $x, y, z$ удовлетворяют условию \\ $\frac{1}{x + y + z} = \frac{1}{x} + \frac{1}{y} + \frac{1}{z}$. Верно ли, что среди них есть два противоположных?

\q6. $2020$ алкоголиков $2021$ раз соображали на троих, причем никакая тройка не повторялась дважды. Докажите, что найдутся две тройки, которые пересекаются ровно по одному алкоголику.

\q70. Окружность $\Omega$ пересекает стороны $BC, CA, AB$ треугольника $ABC$ в точках $A_1, A_2; B_1, B_2$ и $C_1, C_2$ соответственно. Оказалось, что прямые $AA_1, BB_1, CC_1$ пересекаются в одной точке. Докажие, что тогда и прямые $AA_2, BB_2, CC_2$ также пересекаются в одной точке.


\end{document}\documentclass{article}